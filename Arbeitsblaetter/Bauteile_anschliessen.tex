\documentclass[11pt, a4paper]{article}
\usepackage[dvips, bottom=2.5cm, top=2.5cm, right=2.5cm, left=3cm]{geometry}
\usepackage[utf8]{inputenc}
\usepackage[ngerman]{babel}
\usepackage[babel,german=quotes]{csquotes}
\usepackage[T1]{fontenc}
\usepackage{enumerate}
\usepackage{setspace}
\usepackage{mdframed}
\setstretch{1.3} 
\setlength\parindent{0pt}
\usepackage{textcomp}


\begin{document}
%\pagestyle{empty}

\section*{Anschluss von Bauteilen an den Raspberry Pi}

Am Raspberry Pi können verschiedene Bauteile für die Objekte angeschlossen werden. Dazu verfügt jedes Bauteil über Anschlüsse, die auf dem Steckbrett mit dem GPIO des Raspberry Pi verbunden werden müssen. Dabei ist es wichtig, die Kabel so zu stecken, wie es die Anschlüsse auf dem Bauteil vorschreiben.\\

Immer wenn auf einem Bauteil ">PIN"< aufgedruckt ist, kann am GPIO ein beliebiger ">IO"<-Pin ausgewählt werden, der dann später im Programm angegeben werden muss (zum Beispiel um das Bauteil zu steuern oder den Zustand auszulesen).\\

In Abbildung \ref{fig:bauteile_steckbrett} werden zwei Beispiele gegeben, wie die Bauteile anzuschließen sind.

\begin{figure*}[htb]
    \centering
    %\fbox{
     \subfloat[Anschluss des RGB-Scheinwerfers (Anschluss an ">IO02"<, ">IO03"<, ">IO04"< und ">GND"<).]{\includegraphics[width=7.5cm]{img/anschluss_rgb_led}}
     ~
       \subfloat[Anschluss eines Helligkeitssensors (Anschluss an ">IO03"<, ">3V3"< und ">GND"<).]{\includegraphics[width=7.5cm]{img/anschluss_phototransistor}}
    \caption{Angeschlossene Bauteile am Steckbrett.}
    \label{fig:bauteile_steckbrett}
\end{figure*}

Benötigt ein Bauteil einen Anschluss für ">+3,3V"< oder ">GND"<, kann ein beliebiger ">3V3"<- oder ">GND"<-Pin ausgewählt werden. An die ">3V3"<- oder ">GND"<-Pinne können dabei mehrere Bauteile angeschlossen werden, an die ">IO"<-Pinne allerdings nicht. 


%    Alternativ zu den Bildern mit einer Tabelle:
% 
% Abstrakte Tabelle zum Anschluss eines Lichtsensors (Phototransistor)
%\begin{table}[htbp]
%  \centering
%  \begin{tabular}{@{} ccc @{}}
%    \toprule
%    Lichtsensor &  & GPIO \\ 
%    \midrule
%    PIN & $\qquad\longleftrightarrow\qquad$ & IO11 \\ 
%    \textcolor{red}{+3,3V} & $\longleftrightarrow$ & 3V3 \\ 
%    \textcolor{blue}{GND} & $\longleftrightarrow$ & GND \\ 
%    \bottomrule
%  \end{tabular}
%  \caption{Anschluss eines Lichtsensors.}
%  \label{table:anschluss_lichtsensor}
%\end{table}

\subsection*{Anschließen und Anschalten eines Bauteils}
Zum Anschließen und Anschalten eines Bauteils geht man so vor:

\begin{enumerate}

\item Raspberry Pi ausschalten.

\item Das Bauteil mit Jumper-Kabeln mit dem GPIO des Raspberry Pi verbinden (die blaue Seite am Flachbandkabel muss zum Displayanschluss zeigen). Dazu ist auf jedem Bauteil vorgegeben, wie die Kabel gesteckt werden müssen. \textbf{Bitte noch einmal vergewissern, dass das Bauteil und der GPIO richtig angeschlossen wurden!}

\item Raspberry Pi einschalten (und ggfs. einloggen).

\item GroovyConsole starten: Dazu muss das Terminal gestartet werden und in den Ordner der Installation der ">rpCollection"< navigiert werden. Dies geschieht mit dem Befehl \bashinline{cd} (change directory), z.\,B.:

\lstset{language=Bash}
\bashstyleoneline 
\begin{lstlisting}
cd Desktop/rpCollection
\end{lstlisting}

Dort muss dann mit dem Befehl 
\begin{lstlisting}
git pull
\end{lstlisting}
die neuste Version der ">rpCollection"< heruntergeladen werden.

Danach kann mit dem Befehl
\lstset{language=Bash}
\bashstyleoneline
\begin{lstlisting}
bash start.sh
\end{lstlisting}
die GroovyConsole gestartet werden. Dies kann ein paar Sekunden dauern.

\item Nun kann mit den Bauteilen gearbeitet werden: In Quelltext \ref{listing:einfache_led} ist ein Beispiel gegeben, wie die Befehle zum Schalten eines Scheinwerfers (einer LED) lauten.

\textbf{Hinweis:} Mit dem Befehl
\lstset{language=Java}
\javastyleoneline
\begin{lstlisting}
Thread.sleep(Zeit in Millisekunden);
\end{lstlisting}
">wartet"< das Programm die angegebene Zeitspanne.

\item Sollte einmal unklar sein, wie eine Methode bei einem Bauteil heißt, gibt es eine Dokumentation im Ordner ">Dokumentation"< im Installationsordner der ">rpCollection"<. Diese wird über die Datei \texttt{index.html} geöffnet. In der linken Seitenleiste können dann z.\,B. alle Informationen zum RGBScheinwerfer aufgerufen werden.  

\end{enumerate}

\vspace{0.2cm}
\lstset{language=Java}
\javastyle 
\begin{lstlisting}[caption={Steuerung eines Scheinwerfers.},label={listing:einfache_led}]
// Dies ist ein Kommentar. Er wird nicht weiter beachtet.

// Erzeuge das Objekt 'gelbeLampe' fuer die LED, die an Pin 11 (IO11) angeschlossen ist:
gelbeLampe = new Scheinwerfer(11);

// Rufe die Methode 'anschalten()' beim Objekt 'gelbeLampe' auf
gelbeLampe.anschalten();

// Setze den Attributwert 'Martins Buehne' fuer das Attribut 'standort' fuer das Objekt 'gelbeLampe' (Zeichenketten sind in Anfuehrungszeichen zu setzen):
gelbeLampe.setzeStandort("Martins Buehne");
\end{lstlisting}

% 	WEITERE HINWEISE:
%
%// Der Aufruf funktioniert immer so: 
%// <OBJEKTNAME>.<METHODENNAME>
%// Achtung: Die runden Klammern beim Methodennamen duerfen nicht vergessen werden.
%// Hat eine Methode einen Parameter, so muss der Parameterwert in den runden Klammern angegeben werden. Mehrere Parameterwerte sind durch Kommata zu trennen (auf die Reihenfolge achten!).


\vfill
\subsection*{Aufgabe}
\begin{enumerate}
\item Modelliere den Ablauf aus der Problembeschreibung aus der Sicht von Martin.
\end{enumerate}






% Folgendes evtl. in eigene Datei auslagern:
%
%     Tabelle des GPIO-Controllers
%
%\newpage
%\begin{table}[htb]
%\centering
%\scalebox{1}{
%\begin{tabular}{>{\columncolor[gray]{1}}ll|>{\columncolor[gray]{0.9}}c >{\columncolor[gray]{0.9}}c|rr}
%\multicolumn{1}{l}{} & \multicolumn{1}{l}{\textbf{BCM}} & \multicolumn{1}{c}{}  & \multicolumn{1}{c}{}  & \multicolumn{1}{r}{\textbf{BCM}} & \multicolumn{1}{r}{}   
%\vspace{3mm}\\
%\arrayrulecolor[gray]{0}
%\hhline{>{\arrayrulecolor[gray]{1}}-->{\arrayrulecolor[gray]{0}}|--|>{\arrayrulecolor[gray]{1}}--|}
%\arrayrulecolor[gray]{0}
%& 3V3 & $\bullet$ & $\bullet$ & 5V &  \\
%& IO2 & $\bullet$ & $\bullet$ & 5V & \\
%& IO3 & $\bullet$ & $\bullet$ & GND & \\
%& IO4 & $\bullet$ & $\bullet$ & IO14 &  \\
%& GND & $\bullet$ & $\bullet$ & IO15 &  \\
%& IO17 & $\bullet$ & $\bullet$ & IO18 &  \\
%& IO27 & $\bullet$ & $\bullet$ & GND &  \\
%& IO22 & $\bullet$ & $\bullet$ & IO23 &  \\
%& 3V3 & $\bullet$ & $\bullet$ & IO24 &  \\
%& IO10 & $\bullet$ & $\bullet$ & GND &  \\
%& IO9 & $\bullet$ & $\bullet$ & IO25 &  \\
%& IO11 & $\bullet$ & $\bullet$ & IO8 &  \\
%& GND & $\bullet$ & $\bullet$ & IO7 &  \\
%& ID.SD & $\bullet$ & $\bullet$ & ID.SC &  \\
%& IO5 & $\bullet$ & $\bullet$ & GND &  \\
%& IO6 & $\bullet$ & $\bullet$ & IO12 &  \\
%& IO13 & $\bullet$ & $\bullet$ & GND &  \\
%& IO19 & $\bullet$ & $\bullet$ & IO16 &  \\
%& IO26 & $\bullet$ & $\bullet$ & IO20 & \\
%& GND & $\bullet$ & $\bullet$ & IO21 & \\
%\hhline{--|>{\arrayrulecolor[gray]{0.9}}-->{\arrayrulecolor{black}}|--|}
%\rowcolor[gray]{0.9}
%\multicolumn{1}{|c}{} & \multicolumn{1}{c}{}  & \multicolumn{1}{c}{} & \multicolumn{1}{c}{} & \multicolumn{1}{c}{} & \multicolumn{1}{c|}{} \\
%\hline
%\end{tabular}
%}
%\caption{Pin-Belegung auf dem GPIO (BCM-Layout).}
%\label{table:pinb_legung_bcm}
%\end{table}

\end{document}