\documentclass[11pt, a4paper]{article}
\usepackage[dvips, bottom=2.5cm, top=2.5cm, right=2.5cm, left=3cm]{geometry}
\usepackage[utf8]{inputenc}
\usepackage[ngerman]{babel}
\usepackage[babel,german=quotes]{csquotes}
\usepackage[T1]{fontenc}
\usepackage{enumerate}
\usepackage{setspace}
\usepackage{mdframed}
\setstretch{1.3} 
\setlength\parindent{0pt}
\usepackage{textcomp}

\begin{document}
\textbf{{\Large Kranwagen-Projekt}}\\


Es soll ein Kranwagen modelliert werden. Der Kran soll sich frei bewegen k�nnen (360\textdegree~ drehen) und den Arm heben k�nnen. Auch die Seilwinde soll sich automatisch drehen.\\

Es gibt verschiedene Leuchten am Kran, die einzuschalten sind, wenn dieser arbeitet (Arbeitsschutz).\\

Damit der Kran keine Arbeiter verletzt, soll beim Drehen ein Sound ert�nen, wie man es von gro�en Fahrzeugen kennt, wenn diese r�ckw�rts fahren. Zudem sollen die Lampen blitzen.\\

Der Kran soll etwas anheben und muss gesteuert werden:

\begin{mdframed}
Aus Sicherheitsgr�nden soll, bevor irgendetwas anders passiert, die Kranbeleuchtung angeschaltet werden. Dann soll der Kran zweimal kurz hupen, um auch akustisch auf sich aufmerksam zu machen. Danach kann die Arbeit beginnen!\\

Der Kran soll einen Container anheben, steht aber noch in der falschen Position. Zun�chst muss er sich f�r 5 Sekunden nach links drehen. Dazu muss der entsprechende Motor angeschaltet werden und dann f�r 5 Sekunden laufen. Danach muss der Kran den Arm etwas senken, um den Container erreichen zu k�nnen. Der entsprechende Motor muss zun�chst wieder angeschaltet werden und dann f�r 2 Sekunden laufen. Dann kann der Motor f�r die Seilwinde angeschaltet werden und ...
\end{mdframed}


\vspace{0.5cm}
\textbf{Aufgaben}:
\begin{enumerate}
\item Entwerfen Sie zu der gegebenen Problembeschreibung mit Hilfe des Verfahrens von Abbott ein objektorientiertes Modell, indem Sie die relevanten Objekte mit ihren Attributen und Methoden identifizieren. Notieren Sie die Objekte als Objektkarten.

\item �berlegen Sie sich eine Komfortable Steuerung f�r die Anforderungen. 

\item \emph{F�r Fortgeschrittene}: Lie�e sich auch ein Not-Aus-Schalter realisieren? Wenn ja, wie?
\end{enumerate}

\vfill
\emph{Bauteile}:
\begin{itemize}
\item \emph{alle}, insbesondere Motoren
\end{itemize}

\end{document}