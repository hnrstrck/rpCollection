\documentclass[11pt, a4paper]{scrartcl}
\usepackage[
	typ=ab,
	fach=Informatik,
]{schule}
\usepackage{mdframed}
\usepackage{textcomp}

\title{Kranwagen-Projekt}

\begin{document}
\section*{Kranwagen-Projekt}

Es soll ein Hafenkran modelliert werden. Der Kran soll sich frei bewegen können (360\textdegree~drehen) und den Arm heben können. Auch die Seilwinde soll sich automatisch drehen.\\

Es gibt verschiedene Leuchten am Kran, die einzuschalten sind, wenn dieser arbeitet (Arbeitsschutz).\\

Damit der Kran keine Arbeiter verletzt, soll beim Drehen ein Geräusch ertönen, wie man es von großen Fahrzeugen kennt, wenn diese rückwärts fahren. Zudem sollen die Lampen blitzen.\\

Der Kran soll etwas anheben und muss gesteuert werden:

\begin{mdframed}
Aus Sicherheitsgründen soll, bevor irgendetwas anders passiert, die Kranbeleuchtung angeschaltet werden. Dann soll der Kran zweimal kurz hupen, um auch akustisch auf sich aufmerksam zu machen. Danach kann die Arbeit beginnen!\\

Der Kran soll einen Container anheben, steht aber noch in der falschen Position. Zunächst muss er sich für 5 Sekunden nach links drehen. Dazu muss der entsprechende Motor angeschaltet werden und dann für 5 Sekunden laufen. Danach muss der Kran den Arm etwas senken, um den Container erreichen zu können. Der entsprechende Motor muss zunächst wieder angeschaltet werden und dann für 2 Sekunden laufen. Dann kann der Motor für die Seilwinde angeschaltet werden und ...
\end{mdframed}

\vfill

\begin{aufgabe}
Entwerfen Sie zu der gegebenen Problembeschreibung mit Hilfe des Verfahrens von Abbott ein objektorientiertes Modell, indem Sie die relevanten Objekte mit ihren Attributen und Methoden identifizieren. Notieren Sie die Objekte als Objektkarten.
\end{aufgabe}

\begin{aufgabe}
Überlegen Sie sich eine Komfortable Steuerung für die Anforderungen. 
\end{aufgabe}

\begin{aufgabe}
\emph{Für Fortgeschrittene}: Ließe sich auch ein Not-Aus-Schalter realisieren? Wenn ja, wie?
\end{aufgabe}

\end{document}